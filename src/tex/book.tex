% Loving The Prophet
\documentclass[twoside,openright]{book}

\usepackage{afterpage}
\usepackage[toc,page]{appendix}
\usepackage{cite}
\usepackage{xunicode,xltxtra}
\usepackage{changepage}
\usepackage{csquotes}
\usepackage{enumitem}
\usepackage{fancyhdr}          % Customized page headers.
\usepackage{fontspec}
\usepackage[hidelinks,
            pdfpagelayout=TwoPageRight,
            pdftitle="A\ Profile\ of\ Dir\ with\ Agricultural\ Background",
            pdfauthor="Mohammad\ Faheem",
            pdfcreator="XeLaTeX"]{hyperref}
\usepackage{longtable}
\usepackage{ltxtable}
\usepackage{multicol}
\usepackage{paracol}
\usepackage{pgfplots}           % Required for inserting graphs
\usepackage{polyglossia}
\usepackage{url}
\usepackage{geometry}
\usepackage{tikz}
\usetikzlibrary{calc}
\usepackage{titling}
\usepackage{titlesec}
\usepackage{xcolor}
\usepackage{booktabs} % To thicken table lines
\usepackage{bidi}% this should be the last package to load.

% Build with: xelatex particles.tex
% Spell check files with: aspell check --mode=tex <filename>

% Allow insertion of a blank page.
\newcommand\blankpage{%
    \null
    \thispagestyle{empty}%
    \addtocounter{page}{-1}%
    \newpage}

% Line breaking options.
% The tolerant environment allows the tolerance to be adjusted for a range of
% paragraphs. For example:
%
% \begin{tolerant}{600}
%   <paragraph content>
% \end{tolerant}
%
\emergencystretch=5pt
%\tolerance=600
\newenvironment{tolerant}[1]{%
  \par\tolerance=#1\relax
}{%
  \par
}

\titleformat{\chapter}[display]
  {\normalfont\Large\bfseries}
  {\chaptertitlename\ \thechapter}{20pt}{\LARGE}

\titlespacing*{\chapter}{0pt}{24pt}{24pt}

\titleformat{\section}[hang]
  {\normalfont\large\bfseries}
  {}{0pt}{\large}

\setmainlanguage{english}
\setotherlanguage{arabic}
\geometry{
  paperwidth=5.5in,
  paperheight=8.25in,
  outer=0.75in,
  inner=0.75in,
  top=20mm,
}

% Setup the page headers using fancyhdr
\pagestyle{fancy}
\fancyhead[OR,LE]{\thepage}
\fancyhead[ER]{\leftmark}
\fancyhead[OL]{\rightmark}
\renewcommand{\headrulewidth}{0pt}
\renewcommand{\chaptermark}[1]{%
    \markboth{\textsc{#1}}{}}
\renewcommand{\sectionmark}[1]{\markright{\textsc{#1}}}

% There are no footers.
\fancyfoot{}

% Use Garamond in the title, to add interest
\newfontfamily\titlefont[Scale=1.5]{Cormorant Garamond}

% Use Go for emails and URLs
\newfontfamily\emailfont[Scale=0.85]{Go Mono}

%\newfontfamily\arabicfont[Script=Arabic,Scale=1.2]{KFGQPC Uthmanic Script HAFS}
\newfontfamily\arabicfont[Script=Arabic,Scale=1.2]{KFGQPC Uthman Taha Naskh}
%\newfontfamily\arabicfont[Script=Arabic,Scale=1.2]{Scheherazade}
\let\ayah\textsuperscript
\def\diptote{\textsuperscript{2}}

% A transliterated word.
\newcommand{\transl}[1]{\emph{#1}}

% Sallalahu-alayhi-wasallam
\newcommand{\saws}{~\textarabic{ﷺ}}

% A prophetic saying
\newcommand{\hadith}[1]{\tsmallcaps #1}
\def\tsmallcaps #1 #2 {\textsc{#1 #2 }}

% A reference to a book
\newcommand{\bookref}[1]{\emph{#1}}

% This is the relevant environment
\newenvironment{CenteredLastLine}%
   {\setlength{\leftskip}{0pt plus.5fil}\setlength{\rightskip}{0pt plus-.5fil}\setlength{\parfillskip}{0pt plus1fil}}{\par}

% An Ayah with translation
\newcommand{\tayah}[2]{
\begin{quote}
\begin{center}\begin{Arabic}#1\end{Arabic}\end{center}
\vskip 12pt
\begin{CenteredLastLine}
#2
\end{CenteredLastLine}
\end{quote}
}

% Sections do not show a leading section numbers.
% https://tex.stackexchange.com/questions/136527/section-numbering-without-numbers
\makeatletter
% we use \prefix@<level> only if it is defined
\renewcommand{\@seccntformat}[1]{%
  \ifcsname prefix@#1\endcsname
    \csname prefix@#1\endcsname
  \else
    \csname the#1\endcsname\quad
  \fi}
% define \prefix@section
\newcommand\prefix@section{}
\makeatother

% Define the \rom command to allow easy type-setting of Roman numerals
\makeatletter
\newcommand*{\rom}[1]{\expandafter\@slowromancap\romannumeral #1@}
\makeatother

% Don't allow the last line of a paragraph onto a new page. Similarly, do not
% leave the first line of a paragraph all alone on the previous page.
\widowpenalties 2 1000 0
\clubpenalties 2 1000 0
\raggedbottom

\title{Loving the Prophet}
\author{}
\date{}

\begin{document}
\pagenumbering{gobble}
\begin{titlingpage}
\let\cleardoublepage\clearpage
\calccentering{\unitlength}{-0.7cm}
\begin{adjustwidth*}{\unitlength}{-\unitlength}     % Adjust center
    \begin{adjustwidth}{-1cm}{-1cm}                 % Extra large front page
      %\begin{tikzpicture}[remember picture, overlay]
      %  \draw[line width = 2pt] ($(current page.north west) + (5mm,-5mm)$) rectangle ($(current page.south east) + (-5mm,5mm)$);
      %\end{tikzpicture}
      \begin{center}
      {\textbf{\titlefont{\textsc{\Large{A\ Profile\ of\ Dir\ with\ Agricultural\ Background}}}}} \\
      \vspace{0.25cm}
      {\Large{\begin{Arabic}صلى الله عليه وسلم\end{Arabic}}}

      \vspace{2.8cm}
      \includegraphics[width=3in]{./cover.eps}
      \vfill
      \textsc{\Large{By Mohammad Faheem}}

      \vspace{5mm}
      \textit{\large{Translated from the original Arabic}}

      \end{center}
    \end{adjustwidth}
\end{adjustwidth*}
\end{titlingpage}

\begin{titlingpage}
\let\cleardoublepage\clearpage
\vspace*{\fill}

\noindent \textsc{Muslim American Society} \\
712 H Street NE, Suite 1258 \\
Washington, DC 20002 \\
{\emailfont muslimamericansociety.org}

\vspace{12mm}

\small{
\noindent Copyright © 2019 by \textsc{Muslim American Society (MAS)}. \\
All rights reserved. No part of this publication may be reproduced, stored in a
retrieval system or transmitted in any form or by any means, electronic,
mechanical, photocopying, recording or otherwise, without the prior written
permission of MAS.

\vspace{3mm}
\noindent Published by MAS Publications, Washington, DC.
\vspace{5mm}

\noindent Interior design by Amin Ahmad ({\emailfont amin.ahmad@gmail.com}). \\
Cover design by Mustafa Hannigan. \\
Printed in the U.S.A. \\
\vspace{2mm}

\noindent ISBN 978-0-9792113-0-0
}
\end{titlingpage}

\setcounter{page}{3}
\tableofcontents

\mainmatter
\chapter{Preface}

It is a three decade period that Dir District, until recently a state, has been
in the pace of development. Many agencies have been working in this very
backward area. They include regular Govt. Departments, the local councils, the
Rural Development Department and others. Alongside constructive works,
destructive activities have also been there which have caused a great damage and
there is a greater need now for curative measures. References would be made to
such subjects in the coming pages.

Nonetheless, much more has been done and a good step forward has been taken in
the economic development of the area. If a man who left this district two
decades ago, is again visiting the area, he would find it quite a different
environment. He would be seeing quick means of transport, some good pieces of
road, good trade and business and above all a net work of educational
institutions in the area. One cannot shun off the memory of the tough journey by
road when one could travel from Dir to Chakdara in not less than nine long hours
in the buses of old fashion.

We find that there is a net work of feeder roads even to small villages up in
the mountains and the toyota pick-ups have brought a revolution in getting at
these distant point which previously could only be reached after a day long toil
and tough footwalk. There has been quite a visible development in the
infrastructure and we see villages lit up with electricity every where. There
are suspension bridges, roads, schools, hospital, veterinary centres, offices
and every thing related to the concept of modern development. Inspite of all
these, there is still a great need for accelerated development and much more
have to be done for the betterment of the masses specially the small farmers.

One of the negative development in this district is the increased production of
poppy during the past few years. It has got its own reasons and this crop has
now become a need of the people specially in the valleys of Sultankhel, Usherai,
Niag, Tormang and Karo to whom a new name as Eastern Valleys has been given.
Some people have argued that poppy cultivation is a long old tradition of the
area. This is an incorrect statement. Poppy has never been a cash crop of the
area specially during the reign of Nawab, the late Shah Jehan who ruled the area
for 35 long years (1925--1960). He had a very stern fist and would not allow
anybody to go for poppy. However growing of few plants for domestic or medicinal
purposes were not uncommon.

With the advent of new era, Dir saw three different periods in which the focus
of attention were three income sources or fields. A period of timber business
when the  ex-kingdom was finished in 1960. With the introduction of Forest
Department in the  new set up, the forest business became the main source of
income and this was the most lucrative field for many people. The  first forest
road was opened connecting Dir and Kohistan. There hummed up the contractor, the
forestman, the capitalist and every one else and the destruction of forest trees
started. Some people became  millionaires at the cost of the reserved forest
wealth. After some years the game slowed down and there was general overturn
towards the Middle-East and Gulf states in order to earn more and to keep a
better standard. People used every thing possible to get into the Gulf States.
Even fake visas were used for this purpose.

The peak was the seventies and early eighties when people managed to go out in
great number to the Arab states. They did not invest their hard earned money in
any profitable business but started building pacca houses to be furnished with
modern amenities. They found little jobs on return and had to search for new
sounce of income. They could find none except the ``Drug Business''.

There was a heavy demand for the white powder and raw opium and the business
thrived well. On the basis of the ``supply and demand'' the cultivated area
increased under the poppy crop and it took a heavy toll when the attention of
all the people was diverted to its alarming proportions. The west was greatly
affected as the drug addiction there increased at an alarming rate. They had to
do something to curb the evil. The current strategy is to strengthen enforcement
on the trafficking. But on the production side the policy is to develop the
areas through the development process to such an extent that income starts
accruing from income generating schemes.

The scope of this work is very limited and cannot encompass the whole past,
present and future of the district with reference to the development process.
However, a little can be mentioned of the green sector in an elaborated form
since the development of the area is directly dependent on Agriculture as more
than 85\% of the population is connected in one way or the other to this
profession.

The story of Dir remains incomplete when it is separated from that of the ruling
dynasty of the ex-state. There is very little information available about the
history of the ex-state. It is for that reason that a small account will be
given of the ex-ruling dynasty so as to have a background of the area. It is
specially important for those who are engaged in one way or another, in the
development works of Dir that they should know a little bit of the historical
background of the district.

We are thankful to all the officers and functionaries of the Nation Building
Departments, specially officers of the Agriculture, Livestock, FDC and staff of
the DDDP, who helped us by supplying necessary information on different items.
Our thanks are also due to Mr. V. Venturello the Officer in charge, TSU who
encouraged us and offered facilities for getting this booklet released. Mr.
Jehangi Khan, Mr. Gul Wali Khan, Mr. Sultan Rome Mr. Riaz and other of the TSU
deserves our special thanks for their good work of getting the manuscript typed
and computerized.

We Mohammad Fahim and Abu Saeed hope that this small book will be of special
interest to those engaged in the development activities in Dir District.


\chapter{Dir}
\section{Introduction}

Dir is one of the backward districts of the N.W.F.P., situated in the North West
of the Province. It joins Afghanistan and Bajaur on its western side. In the
North it is separated from Chitral by the famous Lowari top. In the east it goes
parallel to the district of Swat and in the south it joins Malakand Agency. The
importance of the district further increase when it is seen through the angle of
its peculiar geographical situation. It has been bestowed with natural resources
of forest, water and fertile soil. There is every possibility of precious mines
wealth if exploited.

The total reported area is 269,200 hectare. The cultivable
area has been estimated to be 95,100 hectare with a culturable
waste land reported to be 1,420 hectare. Out of the total
cultivable area about 59,000 hectare are irrigated. Area under
forest has been reported as 162,400 hectare. The elevation range
from 650 to 4,500 meters. The range of humidity has been
calculated as 59.9 the lowest and the highest as 73.42, with 63.35
as average. The annual precipitation is 700 mm with the
temperature range from $-2$°C to 38°C.

The altitude of some of the important places are:

\begin{center}
\begin{tabular}{ l r l }
 Chakdara & 700 M &  \\
 Timergara & 900 M &  \\
 Wari & 1,000 M & \\
 Dir & 1,500 M & \\
 Lowari & 3,500 M & (the pass is 3,118 M) \\
 Sharingal & 1,200 M & \\
 Thal & 1,800 M & \\
 Kumrat & 2,200 M & \\
 Tarpatar & 1,400 M & \\
 Gorkooi & 2,200 M & \\
 Palam & 1,600 M & \\
 Samar Bagh & 1,000 M & \\
\end{tabular}
\end{center}

Population shown on record is 769,000. It seems to be doubtful. According to a
recent reliable estimation at present the population may be as high as 1.1
Million (the 1991 census figures are yet to be published). Administratively Dir
has been divided into 4 Sub Divisions, 11 Tehsils and 50 Union Councils. (This
has been shown on the chart No.1). The number of farm families have been
estimated as 127,250.

The average holding size with \% age is given as below:--

\begin{center}
\begin{tabular}{ l l c r }
1. & Under 1.00 & Acres & 20\% \\
2. & 1 to 2.5 & " & 51\% \\
3. & 2.5 to 5 & " & 16\% \\
4. & 5 to 7.5 & " & 7\% \\
5. & 7.5 to 12.5 & " & 4\% \\
6. & 12.5 to 25 & " & 2\% \\
7. & \multicolumn{3}{l}{25 to 50 Acres and above Very nominal} \\
\end{tabular}
\end{center}


The land tenure situation of the district is:


\begin{center}
\begin{tabular}{ l l l }
 Pure Tenants & 19.8 & \% \\
 Tenants owing lands & 21.41& \% \\
 Owners operators & 58.78 & \% \\
\end{tabular}
\end{center}

Area under major crops estimated in hectare is as follow. (These figures are
based quite on estimation as no regular revenue system is existing.)

\begin{center}
\begin{tabular}{ l l r r r }
\hline
 & Crop & Irrig Area & Un-irrig Area & Total \\
\hline
1. & Wheat & 17,275 & 22,418 & 39,693 \\
2. & Rape & --- & 6,760 & 6,760 \\
3. & Maize & 9,460 & 5,190 & 14,650 \\
4. & Rice & 15,808 & --- & 15,808 \\
5. & Onion & 2,400 & --- & 2,400 \\
6. & Tomato & 1,300 & --- & 1,300 \\
7. & Poppy & 2,500 & 300 & 2,800 \\
8. & Moong/Mash & 2,800 & --- & 2,800 \\
9. & Beans & 600 & --- & 600 \\
10. & Potato & 1,525 & --- & 1,525 \\
11. & Barley & 500 & 7,000 & 7,500 \\
12. & Deciduous & & & \\
 & Fruit Orchards & 2,000 & --- & 2,000 \\
13. & Citrus Orchards & 500 & --- & 500 \\
14. & Walnut Trees & --- & --- & 180,000 \\
\hline
\end{tabular}
\end{center}

\section{Climate and Natural Vegetation}

The climatic conditions of the district show great variation
due to different physiographic features specially altitude.
Different types of vegetation is seen throughout the area. The
area can be divided into 4 different sub climatological zones which
are briefly discussed below:--

\subsection{Sub Humid Tropical Zone}

The south eastern and south western parts comprising the areas of Timergara,
Talash, Maidan and Jandool with their mountainous belt comprises this part where
the altitude range is 600 to 1,500 meters with annual precipitation range from
500--1,000 mm. The mean temperature in this zone remains above 10°C. for more
than 8 months of the year. The natural vegetation in this zone includes mainly
quercus spp (Banj) Olea cuspidata (Kau) Acacia modesta (Paloosa) Dadonaea
Viscosa (Sanatha) and others. At the comparatively high altitude Pinus rox
burgii (Chir) is also found as natural vegetation. Unfortunately this valuable
forest tree has been under the cruel cutting process for the last few decades
and very little replenishment is  done on community basis.

\subsection{Pine Zone}

This zone is at 1,200--2,300 meters with severe cold winters and mild summers.
Pine forest and range land constitute this zone. Natural forest are \emph{Pinus
roxburgqii} (Chir) \emph{Pinua sallichina} (Kail) and pinus species (Chalghoza).
Chir pine usually occurs below 1,700 meters while Kail and Chalghoza above up to
2,300 altitudes. Some Cedarus deodara (Diyar) also occurs on higher altitude.
Only one crop is possible in the upper limits of this zone due to snow falls.
Major crops are wheat, barley, maize and potatoes. This zone has good potential
for off season vegetables.

\subsection{Fir Zone}

It is steep mountainous zone 2,300--3,200 meters altitude with severe winters
and pleasant cool summers. Natural vegetation includes \emph{Cedarus deodara}
(Diyar) which is the main pine up to 2,400 meter. Above this is FIR and broad
leave species likely Banj. Timber production is the characteric of the zone.

\subsection{Alpine Pastures}

This zone is between 3,200--4,800 meters high which remains frozen for the most
part of the year. Only a short freeze free season permits plant growth. There is
some vegetation in the lower part but the higher portions remain naked when
unsheltered by snow covers.

\section{Sub Division/Tehsils Union Councils and Population}

\subsection{Sub Division}

This district has been sub divided into four sub Givisions, administratively
viz, Timergara, Samar Bagh, Wari and Dir.

\subsection{Tehsils}

Each sub division consists of different number of tehsils. Timergara is composed
of four tehsils i.e. Timergara, Balambat. Lal Qilla and Adenzai. Samar Bagh has
got two tehsils, Munda and Samar Bagh. Wari is divided into two tehsils i.e.
Wari and Khal. Dir Sub Division consists of three tehsils of Dir, Barawal ana
Kalkot.

\vspace{5mm}

% https://tex.stackexchange.com/questions/5073/making-a-simple-directory-tree

\begin{tikzpicture}[every node/.style = {shape=rectangle, draw, align=center}]]
  \node {Dir \\ (4 Sub.Div)}
    child {
      node {Timergara}
      child { node {Timergara \\ 3 U.C. } }
      child { node {Balambat \\ 5 U.C. } }
      child { node {Lal Qila \\ 5 U.C. } }
      child { node {Adenzai \\ 5 U.C. } }
    }
    child {
      node {Samar bagh}
      child { node {Munda \\ 3 U.C. } }
      child { node {Samar Bagh \\ 4 U.C. } }
    }
    child {
      node {Wari}
      child { node {Wari \\ 8 U.C. } }
      child { node {Khal \\ 2 U.C. } }
    }
    child {
      node {Dir}
      child { node {Dir \\ 10 U.C. } }
      child { node {Barawal \\ 3 U.C. } }
      child { node {Kalkot \\ 2 U.C. } }
    }
\end{tikzpicture}

\vspace{5mm}
\begin{enumerate}
  \item Total Sub Divisions: 4 Nos
  \item Total Tehsils: 11 Nos
  \item Total Union Councils: 50 Nos
\end{enumerate}


\chapter[From a Kingdom to the status of a District]
        {Dir---Its March from a Small Kingdom to the status of a District%
          \chaptermark{From a Kingdom to the status of a District}}
\chaptermark{From a Kingdom to the status of a District}

\section{Sultan Khel and Painda Khel}

\textsc{The Origin}: Tarkan and Yousaf were two brothers, originally belonging
to Ziada (Sawabi) in the 13 century A.D. The descendants of Tarkan are called as
Tarkani and the tribes occupying the area of Salarzai, Mamoond and Utman Knel in
Bajaur and the area of Maidan, Jandool and the area of Barawal up to Ganori and
Salamkot in Dir District.

The descendants of Yousaf are called as Yousafzai and they are prevailing in the
areas of Buner, Mardan, Sawabi, Swat, Dir, Sind (the area of Dir District from
Rabat downward on both side the Panjkora river up to the foot of Kamrani) Talash
and Adenzai. In the north Yousafzai tribes are present up to Sharingal Gadgadeen
(the barrier between Kohistan and Sharingal).

Paind whose descendants are called as Painda Khel, is the grand-grand-grand son
of Yousaf. The link goes like this: Painda son of Ghori, son of Kahay, son of
Akoo son of Yousaf. Therefore, they are called Yousafzai.

Painda's father Ghori lived in Nikpi Khel, Swat. After his death his wife
migrated to Kuhna Dher (now called as Haji Abad) a few kilometers ahead of
Balambat. The mother brought this child in her lap. She was later on married to
a man called as Maly Baba. Sultan was the child of Maly Baba from this woman.
Thus Painda and Sultan were brothers from the same mother.

\section{Maly (14\textsuperscript{th} Century)}

Maly is the grand father of Malizai tribe. He was the son of Khwajur, son of
Akoo son of Yousaf. Maly was called as Maly Baba because he was respected in the
area for his religious preaching and also political domination. his influence
was prevalent in Sind, Maidan, Barawal and Kohistan.

Maly Baba had two sons, Awsa Khan and Nusrat Khan from his first wife. A third
son Sultan was born from the newly married widow, the mother of Painda. Maly
after whose name his descendants are called as Malizai tribe. sub divided his
landed property and estate among his sons. He also gave share to his step-son
Painda.

\section{Distribution of the Area and Landed Property}

The area from Akhagram to Koto Kuhna Dher (now called as Haji Abad)
was given to Nusrt Khan. In addition the area westwards up
to Khkuly Gaty Haya Serai, was also added to it. It is after the
name of Nusrat Khan that the people of these areas are called Nuradin Khel.

The territory from Koto to Walai Kandaw, Talash, Sado to Ashari Gat was allotted
to Awsa Khan. Therefore, the people occupying these areas were termed as Awsa
Khel. This area also includes Timergara, Andheri, Balambat and Malakand.

The area from Khal to Khan Kas (Darora) on the right side of river Panjkora was
given to Sultan. In addition, the area including Tormang called as Dalkha Khel
right to Asbanr through the hill-tops was also given to Sultan. The area on the
right side of Panjkora river covering Kotki, Pataw and the lower parts alongside
the Panjkora river were also given to Sultan. Now these areas on the right side
of the river are the property (Dawtar) of Sultan Khel.

Painda was given comparatively less fertile and to some extent barren parts
consisting of un-irrigated patches and rough terrain. It included the area
stretching from Akhagram up to the top peaks of Shalga to Shumai and Jabai, the
farthest boundaries of Usherai Dara which are the barriers with Swat. This was
the domain of Painda now called as Painda Baba. It also included Lar Jam, Urtha
Sind, Atan etc. The area from Khal bazar to Akhagram on the left side of
Panjkora is also the property of Sultan Khel. The area along the NCC road from
Akhagram to Darora (Khan Kas) is the property of Painda Khel.

Alongside these big tribes there also live in these areas some other small
sections of other tribes seeking cover and protection of these big ones. They
include tribes like Utman Khel, Swati, Roghani, Shinwari, Dalazak, Wardag, Gujar
etc. These tribes have now also flourished well alongside the dominant tribes.

\section{Painda and His Sons (Painda= 1500--1565)}

Painda Baba had two sons named Mubarak Khan and Musa Khan. By the time Painda
was intending to subdivide his area of control between his two sons, Mubarak
Khan and Musa Khan, the later died leaving his son Bahamat Khan to succeed him
as legal heir. The grand father was kind enough to allow Bahamat Khan the share
his father would have received had he been alive at the times of property
distribution.


The villages of Dugram, Kot, Mula Gujar, Diskor, Jatgram,
Kakad, Badalai, Niag, Bandai, Ari Manzai, Qadar, Mathorh, Khandar
and Kuwan with their shamilats all in Niag Dara were given to the
grand son Bahamat Khan. (1580--1636)

\section{Akhoond Ilyas (Akhoon Baba) (1626--1708)}

Akhoon Ilyas son of Brahim son of Bahamat Khan son of Musa Khan son of Painda
was a saintly man. He was in search of some ``Murshad'', and finally reached
Sheikh Adam, commonly known as Hazrat Binor. He was preaching religion in Delhi.
This was the time when Shah Jehan was ruling India. The emperor had banned the
entry of Hazrat Binor into Delhi. He cam to Bombay and was given protection by
Crown Prince Aurang Zeb, Alamgir the famous saintly type king of the Mughal
empire. Prince Aurang Zeb Alamgir managed to send Hazrat Binor to Hajj via Adan.

Akhoon Ilyas accompanied Hazrat Binor and performed pilgrimage at Mecca.
Therefrom he returned to his homeland in order to preach religion. The Murshad
gave him a pair of shoes and beeds as a token of sacred belongings which were
kept as sacred legacy by the ruling dynasty in their custody in succession. The
Akhoon became an important religious personality to whom people diverged from
far and wide including the area of Bajaur, Swat and Buner in order to get
spiritual values and exaltation. By that time Dir was inhabited by infidel
people and were called as ``Black Kafirs''.

Prior to this saintly man another Akhoon called Akhoon Salak had also come to
this area from Peshawar in order to preach religion and fight against the
infidelity prevailing in the area.

By the time Akhoon Baba came here two main tribes Painda Khel and Sultan Khel
had flourished well and were the two powerful entities to be reckoned to in this
area.

Akhoon Baba continued his religious call and preaching with zeal and devotion.
This call attracted people from all walks of life and from different places
bringing with them valuables and cash to be offered as gratifications. Similarly
people offered him arable lands as gratitude locally called (Serai or Shukrana).
The dynasty flourished becoming powerful both worldly as well as religiously.
They had a strong hold and dominance over the masses. This status continued in
the sons and grand sons of Akhoon.

\section{Mullah Ismail (1708--1782)}

Akhoon Baba had his son named Mullah Ismail. This man was also a religious
preacher and a man of virtues. He is buried on the opposite side of village
Bibyawar across Panjkora, in the name of "Loe Baba" (Grand fatherly man).

\section{Ghulam Khan (1782--1808)}

Mullah Ismail was succeeded by his son Ghulam Khan who had enough quantity of
worldly belongings including land, weaponry and cash. This man started craving
for attaining the status of a Khan alongside with religious leadership of which
the dynasty was custodian.

\section{Khan Zafar Khan (1808--1814 Chiefdom)}

Zafar Khan was the son of Ghulam Khan who came out as a powerful Khan setting
aside the religion guise and tried to bring under control the neighboring
tribes. He left Bibyawar and settled in Dir as the ruling Khan of Dir. However
he died soon and was buried on the left side of Panjkora near village Bibyawar
in the name of Warooki Baba (Small Baba).

\section{Qasim Khan (Khan Shaheed) 1814--1822}

Qasim Khan was the son of Zafar Khan. He succeeded his father
and became Khan of Dir. His other brothers were Naseem Khan, Zahir
Khan and Bakoo Khan. The Akhoon Khel presently available in
Samkot, Batal, Ker Dara and Rokhan are the sons of these brothers.
% TODO: Is it Rokhan or Rokhar above?

Qasim Khan expanded the ``Khani'' of Dir, consolidated his
position and merged out as an aristocratic ruler of the area. He
married the sister of the Wali of Chitral called as Shah Kator.

\section{Assassination of Qasim Khan}

Qasim Khan had four sons named Ghazan Khan, Muhammad Said Khan, Azad Khan and
Sadullah Khan. Qasim Khan was assassinated by his son Azad Khan. He is buried in
Dir proper in the name of Khan Shaheed, just at the foot of the royal palace
Dir.

\section{Ghazan Khan (1822--1866)}

Qasim Khan was succeeded by his son Ghazan Khan who was from
the wife named Khunza Bibi, the sister of Shah Kator, Wali of
Chitral. Ghazan Khan was ascended to the throne at an age of 17
years. He killed some of his brothers and exiled the rest. He
expanded his domain to Asmar in the west, to Gad in north and in
the south to Sakhakot and in the east to Swat. He ruled for 46
long years.

Ghazan Khan was now a typical ruler Khan and gradually the Babaism (a term
carrying religious connotation) had now started to shapen itself into Khanism
with a stronger fist of rulership or at least with an identity of a tribal
eldership. The ruler was not only respected but he also exercised a self
determined authority of power.

\section{Khan Rahmatullah Khan (1866--1880)}

The nominated successor Rahmatullah Khan took the helm of the affairs after the
death of his father Ghazan Khan. He ruled till his death in 1880. Rahmatullah
Khan generally called locally as Ramatullah Khan is buried in Dir. Rahmatullah
Khan had ten sons; Mohammad Sharif Khan, Sher Mohammad Khan, Hayatullah Khan,
Jamroz Khan, Ashraf Khan, Sikandar Khan, Datullah Khan, Shah Nasim Khan, Dilaram
Khan and Rosham Khan. While still alive Rahmatullah Khan distributed the areas
of his domain amongst his sons.

Mohammad Sharif Khan who was the crown prince and also the Commander-in-Chief of
the army turned against his father and created great problem for him,
instigating different tribes to rise against the ruler. He invited Umara Khan
against his father. Umara Khan captured a number of fortresses of Rahmatullah
Khan in the area of Sind. After the death of Rahmatullah Khan in 1880, Mohammad
Sharif Khan quickly moved from Sharingal and succeeded his father as ruler of
Dir.

\section{Mohammad Sharif Khan (1848--1904) Ruled from 1880--1890 and 1895--1904}

Mohammad Sharif Khan is the man in the series who was formally declared as Nawab
by the Britishers. Mohammad Sharif Khan had escaped an assassination attempt by
his brother Jamroz Khan. He killed Jamroz Khan and his son and began to rule the
State but had a number of adventurous events during his tenure as the Nawab so
much so that he had the sad experience of exile. An account of his life will be
given in the coming pages.

Mohammad Sharif Khan had four sons. Amongst them Mian Gul Jan and Aurangzeb Khan
were from one mother and the other two i.e. Sultan Mohammad Khan and Mohammad
Issa Khan were from other two mothers. Mohammad Sharif Khan divided the state
amongst his sons.

Mian Gul Jan, whose real name was Shah Rawan, was given Munda in Jandool and
Sharingal in Urtha Sind. Sultan Mohammad Khan known as Sangar Khan was given
Sadbar Killi in Jandool and Dog Dara in Dir Tehsil. Mohammad Issa Khan known as
Darory Khan was given Darora, Jughabanj, Manoogay etc. Aurang Zeb who succeeded
Sharif Khan as ruler of the State was also known as "Dumb Nawab", though ill and
paralyzed to some extent, he had all the qualities of a good ruler. He succeeded
his father as ruler of the State.


\section{Nawab Aurang Zeb (Badshah Khan) (1904--1925)}

Born in 1874, Nawab Aurang Zeb ruled the State for 21 years (1904--1925). He had
four sons. Amongst them the two i.e. Shah Jehan and Alam Zeb Khan were from one
mother. She was a maid servant in the house of the Nawab and came from village
Bilachand in Barawal area. This beautiful woman was married by the Nawab Aurang
Zeb Khan. The third son named Bacha or Badshah who was from the daughter of a
famous Malik of Swat called Chaman Malak died at an early age. The fourth son
Bakht Jehan Zeb Khan known as Khan of Timer is still living and commands a very
good reputation. He is very noble man and takes keen interest in the well being
and progress of the area. He has proved himself to be the custodian of good and
positive traditions of the dynasty. His mother was the sister of the famous
Mehtar of Chitral Sir Shujaul Mulk.

Nawab Aurang Zeb declared Shah Jehan as crown prince during his life.

\section{Nawab Sir Shah Jehan Khan (1925--1960)}

Nawab Shah Jehan was declared as legal Nawab of the State in 1925, after his
father's death. The Chief Commissioner of the NWFP through a notification
declared him as the Nawab. In 1926 the Viceroy visited the area and met Shah
Jehan. In 1929 again the Nawab had a meeting with the Viceroy while he was on a
visit to the State. In the year 1930 the Viceroy again paid visit to this part
of the province and went up on to the Kamrani top.

There was tussle between Shah Jehan and Alam Zeb Khan and both had groups of
influential Maliks and Khans on their sides. Finally Nawab Shah Jehan succeeded
in exiling Alam Zeb Khan from the State. He sought refuge in Khar Bajaur. In the
year 1931 an agreement was reached between Nawab Shah Jehan and the Khan of
Khar. Under this agreement the Khan of Khar was now bound not to give refuge to
Alam Zeb and he went to Utman Khel Area.

Nawab Shah Jehan was a man of old traditions. He did not like the people to
receive modern education and kept them ignorant of the latest turn in the
civilization. He did nothing for the welfare of the public and even did not like
people going out in search of knowledge and education. The Govt. of Pakistan had
also to fulfill her obligations to the people of the State. The Nawab was asked
through the Resident Commissioner to allow the welfare institutions like school
and hospital in his state, but he always opposed these things. At last the Govt.
of Pakistan had to take action and dethroned the Nawab on 9 October 1960. He
remained house arrest in Lahore till his death in 1968.


\centerline{✽\quad ✽\quad ✽}


\end{document}
